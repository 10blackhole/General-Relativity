\section{Gravitation}
\subsection{Physics in Curves Spacetime}
The physics of gravitation falls naturally into two peaces: how the gravitational field influence the behavior of matter, and how matter determines the gravitational field. In Newtonian gravity, this two elements consist of the expression for the acceleration of a body in a gravitational potential $\Phi$,
\begin{equation}\label{4.1}
    \vb*{a}=-\nabla\Phi
\end{equation}
and Poisson's equation
\begin{equation*}
    \nabla^2\Phi=4\pi G\rho
\end{equation*}
In general relativity, the analogous statements will describe how the curvature of spacetime acts on matter to manifest itself as gravity, and how energy and momentum influence spacetime to create curvature.

The \textcolor{purple}{Einstein Equivalence Principle (EEP): In small enough regions of spacetime, the laws of physics reduce to those of special relativity; it is impossible to detect the existence of a gravitational fields by means of local experiments}. The EEP arises from idea that gravity is \textit{universal}; it affects all particles in the same way. The identification of spacetime as a curved manifold is supported by the similarity between the undetectability of gravity in local regions and our ability to find locally inertial coordinates ($g_{\hat{\mu}\hat{\nu}}=\eta_{\hat{\mu}\hat{\nu}}, \partial_{\hat{\rho}}g_{\hat{\mu}\hat{\nu}}=0$ at a point $p$) on a manifold.

The \textbf{minimal-coupling principle} generalizes laws of physics. This recipe may be stated as follows:
\begin{enumerate}
    \item Take a law of physics, valid in inertial coordinates in flat spacetime.
    \item Write it in coordinate-invariant (tensorial) form.
    \item Assert that the resulting law remains true in curved spacetime.
\end{enumerate}

\begin{tcolorbox}
As a starightforward example, we can consider the motion of dreely-falling (unaccelerated) particles. In flat space such particles move in straight lines; in equations, this is expressed as the vanishing of the second derivate of the parametrized path $x^\mu(\lambda)$:
\begin{equation}\label{eq:4.3}
    \dv[2]{x^\mu}{\lambda}=0
\end{equation}
We can use the chain rule to write
\begin{equation*}
    \dv[2]{x^\mu}{\lambda}=\dv{x^\nu}{\lambda}\partial_\nu\dv{x^\mu}{\lambda}
\end{equation*}
Generalizing this to curved space--simply replace the partial derivative by a covariant one,
\begin{equation*}
    \dv{x^\nu}{\lambda}\partial_\nu\dv{x^\mu}{\lambda}\quad \to\quad \dv{x^\nu}{\lambda}\nabla_\nu\dv{x^\mu}{\lambda}=\dv[2]{x^\mu}{\lambda} + \tensor{\Gamma}{^\mu_{\rho\sigma}}\dv{x^\rho}{\lambda}\dv{x^\sigma}{\lambda}
\end{equation*}
It is the general-relativistic version of the Newtonian relation (\ref{eq:4.3}) is simply the geodesic equation,
\begin{equation*}
    \dv[2]{x^\mu}{\lambda} + \tensor{\Gamma}{^\mu_{\rho\sigma}}\dv{x^\rho}{\lambda}\dv{x^\sigma}{\lambda}=0
\end{equation*}
In general relativity, therefore, \textcolor{purple}{free particles move along geodesics}; we have mentioned this before.
\end{tcolorbox}

\begin{tcolorbox}
As an evenmore straightforward example, we have the law of energy-momentum conservation in flat spacetime:
\begin{equation*}
    \partial_\mu T^{\mu\nu}=0
\end{equation*}
The appropriate generalization to curved spacetime:
\begin{equation*}
    \nabla_\mu T^{\mu\nu}=0
\end{equation*}
\end{tcolorbox}

We define the Newtonian limit by three requirements: the particles are moving slowly (with respect to the speed of light), the gravitational field is weak (so that it can be considered as a perturbation of flat space), and the field is also static (unchanging with time). Let us see what these assumptions do to the geodesic equation. 

\textcolor{purple}{Moving slowly} means that
\begin{equation*}
    \dv{x^{i}}{\tau}\ll \dv{t}{\tau}
\end{equation*}
so the geodesic equation becomes
\begin{equation}\label{4.10}
    \dv[2]{x^\mu}{\tau}+\Gamma^\mu_{00}\left(\dv{t}{\tau}\right)^2=0
\end{equation}
Since \textcolor{purple}{the field is static} ($\partial_0 g_{\mu\nu}=0$), the relevant Christoffel symbols $\Gamma^\mu_{00}$ simplify:
\begin{align*}
    \Gamma^\mu_{00}&=\frac{1}{2}g^{\mu\lambda}(\partial_0 g_{\lambda 0}+\partial_0 g_{0\lambda} -\partial_\lambda g_{00})\\
    &=-\frac{1}{2}g^{\mu\lambda}\partial_\lambda g_{00}
\end{align*}
Finally, \textcolor{purple}{the weakness of the gravitational field} allows us to decompose the metric into the Minkowski form plus a small perturbation:
\begin{align}\label{4.12}
    g_{\mu\nu}=\eta_{\mu\nu}+h_{\mu\nu},\qquad |h_{\mu\nu}|\ll 1
\end{align}
From the definition of the inverse metric, $g^{\mu\nu}g_{\nu\sigma}=\delta^\mu_\sigma$, we find that to first order in $h$,
\begin{equation}\label{4.13}
    g^{\mu\nu}=\eta^{\mu\nu}-g^{\mu\nu}
\end{equation}
We can use the Minkowski metric to raise and lower indices on an object of any definite order on $h$, since the corrections would only contribute at higher orders.

Putting all together, to first order in $h_{\mu\nu}$ we find
\begin{equation*}
    \Gamma^\mu_{00}=-\frac{1}{2}\eta^{\mu\lambda}\partial_\lambda h_{00}
\end{equation*}
The geodesic equation (\ref{4.10}) is therefore
\begin{equation}\label{4.15}
    \dv[2]{x^\mu}{\tau}=\frac{1}{2}\eta^{\mu\lambda}\partial_\lambda h_{00}\left(\dv{t}{\tau}\right)^2
\end{equation}
Using $\partial_0 h_{00}=0$, the $\mu=0$ component of this is just
\begin{equation*}
    \dv[2]{t}{\tau}=0
\end{equation*}
That is, $\dv*{t}{\tau}$ is constant. The spacelike components of (\ref{4.15}),
\begin{equation*}
    \dv[2]{x^{i}}{\tau}=\frac{1}{2}\left(\dv{t}{\tau}\right)^2\partial_i h_{00}
\end{equation*}
Dividing both sides by $(\dd t/\dd\tau)^2$,
\begin{equation*}
    \dv[2]{x^{i}}{t}=\frac{1}{2}\partial_i h_{00}
\end{equation*}
This begins to look a great deal like Newton's theory of gravitation. In fact, if we compare this equation to (\ref{4.1}),
\begin{equation}\label{4.19}
    h_{00}=-2\Phi
\end{equation}
or
\begin{equation}\label{4.20}
    g_{00}=-(1+2\Phi)
\end{equation}
Therefore, we have shown that the curvature of spacetime is indeed sufficient to describe gravity in the Newtonian limit, as long as the metric takes the form (\ref{4.20}).

\subsection{Einstein's Equation}
Just as Maxwell's equations govern how the electric and magnetic fields responds to charge and currents, Einstein's field equation governs how the metric responds to energy and momentum.

The informal argument begins with the realization that we would like to find an equation that supersedes the Poisson equation for the Newtonian potential:
\begin{equation}\label{4.22}
    \nabla^2\Phi=4\pi G\rho
\end{equation}
We might therefore guess that our new equation will have $T_{\mu\nu}$ set proportional to some tensor, which is second-order in derivatives of the metric,
\begin{equation*}
    [\nabla^2 g]_{\mu\nu}\propto T_{\mu\nu}
\end{equation*}
but of course we want it to be completely tensorial.

The first choice might be to act the d'Alambertian $\Box =\nabla^\mu \nabla_\mu$ on the metric $g_{\mu\nu}$, but this is automatically zero by metric compatibility. Fortunately, there is an obvious quantity which is not zero and is constructed from second derivatives (and first derivatives) of the metric: the Riemann tensor $\tensor{R}{^\rho_{\sigma\mu\nu}}$. It doesn't have the right number of indices, but we can contact it to from the Ricci tensor $R_{\mu\nu}$, which does (and is symmetric to boot).
\begin{equation}\label{4.24}
    R_{\mu\nu}=\kappa T_{\mu\nu}
\end{equation}
There is a problem, unfortunately, with conservation of energy. If we want to preserve
\begin{equation*}
    \nabla^\mu T_{\mu\nu}=0
\end{equation*}
bu (\ref{4.24}) we would have
\begin{equation*}
    \nabla^\mu R_{\mu\nu}=0
\end{equation*}
This is certainly not true in an arbitrary geometry; we have seen from the Bianchi identity that
\begin{equation*}
    \nabla^\mu R_{\mu\nu}=\frac{1}{2}\nabla_\nu R
\end{equation*}
But our proposed field equation implies that $R=\kappa g^{\mu\nu}T_{\mu\nu}=\kappa T$, so taking these together we have
\begin{equation}\label{4.28}
    \nabla_\mu T=0
\end{equation}
The covariant derivative of a scalar is just the partial derivative, so (\ref{4.28}) is telling is that $T$ is constant throughout spacetime. This is highly implausible, since $T=0$ in vacuum while $T\neq 0$ in matter. We have to try hard.

Of course we don't have to try much harder, since we already know of a symmetric ($0,2$) tensor, constructed from the Ricii tensor, which is automatically conserved: the Einstein tensor
\begin{equation*}
    G_{\mu\nu}=R_{\mu\nu}-\frac{1}{2}Rg_{\mu\nu}
\end{equation*}
which always obeys $\nabla^\mu G_{\mu\nu}=0$. We are therefore led to propose
\begin{equation}\label{4.30}
    G_{\mu\nu}=\kappa T_{\mu\nu}
\end{equation}
as a field equation for the metric. This equation satisfies all of the obvious requirements: the right-hand side is a covariant expression of the energy and momentum density in the form of a symmetric and conserved ($0,2$) tensor, while the left-hand side is a symmetric and conserved ($0,2$) tensor constructed from the metric and its first and second derivatives. \textcolor{purple}{Does this equation predict the Poisson equation for the gravitational potential in the Newtonian limit?}
To answer this, note that contracting both sides of (\ref{4.30}) yields (in four dimensions)
\begin{equation}\label{4.31}
    R=-\kappa T
\end{equation}
and using this we can rewrite (\ref{4.30}) as
\begin{equation}\label{4.32}
    R_{\mu\nu}=\kappa \left(T_{\mu\nu}-\frac{1}{2}Tg_{\mu\nu}\right)
\end{equation}
his is the same equation, just written slightly different. We woulds like to see if it predicts Newtonian gravity in the weak-field, time-independent, slowly-moving-particles limit. We consider a perfect-fluid source of energy-momentum, for which
\begin{equation*}
    T_{\mu\nu}=(\rho +p)U_\mu U_\nu+ pg_{\mu\nu}
\end{equation*}
In fact for the Newtonian limit we many neglect the pressure; roughly speaking, the pressure of a body becomes important when its constituent particles are traveling at speeds close to that of light. So we are actually considering the energy-momentum tensor of \textcolor{purple}{dust}\footnote{The \textit{fluid} we are considering is some massive body, such as the Earth or the Sun.}:
\begin{equation*}
    T_{\mu\nu}=\rho U_\mu U_\nu
\end{equation*}
We will work in the fluid rest frame, in which
\begin{equation*}
    U^\mu = (U^0,0,0,0)
\end{equation*}
The timelike component can be fixed by appealing to the normalization condition $g_{\mu\nu}U^\mu U^\nu=-1$. In weak-field limit we write, in accordance with (\ref{4.12}) and (\ref{4.13}),
\begin{align*}
    g_{00}&=-1+h_{00}\\
    g^{00}&=-1-h^{00}
\end{align*}
Then to first order in $h_{\mu\nu}$ we get
\begin{equation*}
    U^0=1+\frac{1}{2}h_{00}
\end{equation*}
So to our level of approximation, we can simply take $U^0=1$, and correspondingly $U_0=-1$. Then
\begin{equation*}
    T_{00}=\rho
\end{equation*}
and all other components vanish. In this limit the rest energy $\rho=T_{00}$ will be much larger than other terms in $T_{\mu\nu}$, so we want to focus on the $\mu=0, \nu=0$ components of (\ref{4.32}). The trace, to lowest nontrivial order, is
\begin{equation*}
    T=g^{00}T_{00}=-T_{00}=-\rho
\end{equation*}
We plug this into the $00$ component of our proposed gravitational field equation (\ref{4.32}), to get
\begin{equation}\label{4.40}
    R_{00}=\frac{1}{2}\kappa \rho
\end{equation}
This is an equation relating derivatives of the metric to the energy density. To find the explicit expression in terms of the metric, we need to evaluate $R_{00}=\tensor{R}{^\lambda_{0\lambda 0}}$. In fact we only need $\tensor{R}{^{i}_{0i0}}$, since $\tensor{R}{^0_{000}}=0$. We have
\begin{equation*}
    \tensor{R}{^{i}_{0j0}}=\partial_j\Gamma^{i}_{00}-\partial_0\Gamma^{i}_{j0}+\Gamma^{i}_{j\lambda}\Gamma^{\lambda}_{00}-\Gamma^{i}_{0\lambda}\Gamma^{\lambda}_{j0}
\end{equation*}
The second term here is a time derivative, which vanishes for static fields. The third and fourth terms are the form $(\Gamma)^2$, and since $\Gamma$ is first-order in the metric perturbation there contribute only at second order, and can be neglected. We are left with $ \tensor{R}{^{i}_{0j0}}=\partial_j\Gamma^{i}_{00}$. From this we get
\begin{align*}
    R_{00}&= \tensor{R}{^{i}_{0i0}}\\
    &=\partial_i\left[\frac{1}{2}g^{i\lambda}(\partial_0g_{\lambda 0}+\partial_0g_{0 \lambda}-\partial_\lambda g_{00})\right]\\
    &=-\frac{1}{2}\delta^{ij}\partial_i\partial_j h_{00}\\
    &=-\frac{1}{2}\nabla^2 h_{00}
\end{align*}
Comparing to (\ref{4.40}), we see that the $00$ component of (\ref{4.30}) in the Newtonian limit predicts
\begin{equation*}
    \nabla^2 h_{00}=-\kappa\rho
\end{equation*}
Since (\ref{4.19}) sets $h_{00}=-2\Phi$, this is precisely the Poisson equation (\ref{4.22}), if we set $\kappa=8\pi G$.

So our guess, (\ref{4.30}), seems to have worked out. With the normalization chosen so as to correctly recover the Newtonian limit, we can present \textbf{Einstein's equation} for general relativity:
\begin{equation}\label{4.44}\marginnote{Einstein's field equation.}
    \boxed{R_{\mu\nu}-\frac{1}{2}Rg_{\mu\nu}=8\pi GT_{\mu\nu}}
\end{equation}
This tell us how the curvature of spacetime reacts to the presence of energy-momentum.

It is sometimes useful to rewrite Einstein's equation in a slightly different form. Following (\ref{4.31}) and (\ref{4.32}), we can take the trace of (\ref{4.44}) to find that $R=-8\pi GT$. Plugging this in and moving the trace term to the right-hand side, we obtain
\begin{equation}\label{4.45}
    \boxed{R_{\mu\nu}=8\pi G\left(T_{\mu\nu}-\frac{1}{2}Tg_{\mu\nu}\right)}
\end{equation}
We will often be interested in the Einstein's equation in vacuum, where $T_{\mu\nu}=0$ (for example, outside a star or planet). Therefore the vacuum Einstein equation is simply
\begin{equation}\label{4.46}\marginnote{Vacuum Einstein equation.}
    \boxed{R_{\mu\nu}=0}
\end{equation}
This is both slightly less formidable, and of considerable physical usefulness.